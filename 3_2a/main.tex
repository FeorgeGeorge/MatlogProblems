\documentclass[10pt]{article}
\pagestyle{plain}
\textwidth=15.5cm
\textheight=24.0cm
\oddsidemargin=0.3cm
\evensidemargin=0.5cm

\usepackage[utf8]{inputenc}
\usepackage[T2A]{fontenc}
\usepackage[russian]{babel}
\usepackage{amssymb,amsmath, amsthm}
\usepackage{graphicx}
\usepackage[a4paper,bmargin=2.5cm]{geometry}

\voffset=0pt
\headheight=0pt
\headsep=0pt

\begin{document}

\def\chap#1#2{\ \\ {\large\bf#1 \ | \ \tt\scshape#2} \par}

\ \vspace{-1cm}

{\bf
\ \\
\Large\centerline{\scshape Матлог 3.2a}
}\normalsize

Задача: исследовать топологию Зарисского.

Окрестность точки $x$~--- это любое множество, содержащее $x$, что его дополнение конечно.

Замкнутые множества в этой топологии~ это все конечные множества и само $\mathbb{R}$.
Поэтому  если $X$ конечно, то $Cl X = X$, иначе $Cl X = \mathbb{R}$.

Если $\neg X$ бесконечно, то дополнение любой его внутренности бесконечно и внутренность $X$ пустая.
В противном случае множество совпадает со своей внутренностью.

Это пространство связно, ведь пересечение любых двух открытых множеств не пусто.

Рассмотрим формулу $A \lor \neg A$, переписанную для топ. пространства: 
\[ Y \cup (\mathbb{R} - Y)^\circ,\]
где $Y$ открытое множество, т.е его дополнение конечно. 
Тогда $(\mathbb{R}- Y)^\circ = \emptyset$.
Значит \[ Y \cup (\mathbb{R} -Y)^\circ = Y\]
и эта топология не является моделью классической логики.
\end{document}