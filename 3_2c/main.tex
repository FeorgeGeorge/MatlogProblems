\documentclass[10pt]{article}
\pagestyle{plain}
\textwidth=15.5cm
\textheight=24.0cm
\oddsidemargin=0.3cm
\evensidemargin=0.5cm

\usepackage[utf8]{inputenc}
\usepackage[T2A]{fontenc}
\usepackage[russian]{babel}
\usepackage{amssymb,amsmath, amsthm}
\usepackage{graphicx}
\usepackage[a4paper,bmargin=2.5cm]{geometry}

\voffset=0pt
\headheight=0pt
\headsep=0pt

\begin{document}

\def\chap#1#2{\ \\ {\large\bf#1 \ | \ \tt\scshape#2} \par}

\ \vspace{-1cm}

{\bf
\ \\
\Large\centerline{\scshape Матлог 3.2с}
}\normalsize

\vspace*{1cm}
Топология бесконечных множеств на $\mathbb{R}$.

Надо изучить: 
\begin{enumerate}
    \item окрестности точек
    \item замкнутые множества
    \item внутренности и замыкания
    \item является ли данная топология моделью классической логики
    \item связность
\end{enumerate}

Решение: 

Это множество не образует топологию. 
Пример: два открытых множества $(-\infty, 0) \cup \{1\}$ и $\{1\} \cup (2, +\infty)$
\end{document}