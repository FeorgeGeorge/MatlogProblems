\documentclass[10pt]{article}
\pagestyle{plain}
\textwidth=15.5cm
\textheight=24.0cm
\oddsidemargin=0.3cm
\evensidemargin=0.5cm

\usepackage[utf8]{inputenc}
\usepackage[T2A]{fontenc}
\usepackage[russian]{babel}
\usepackage{amssymb,amsmath, amsthm}
\usepackage{graphicx}
\usepackage[a4paper,bmargin=2.5cm]{geometry}

\voffset=0pt
\headheight=0pt
\headsep=0pt

\begin{document}

\def\chap#1#2{\ \\ {\large\bf#1 \ | \ \tt\scshape#2} \par}

\ \vspace{-1cm}

{\bf
\ \\
\Large\centerline{\scshape Матлог 3.2b}
}\normalsize

\vspace*{1cm}
Топология стрелки на $\mathbb{R}$: $\Omega = \{\emptyset\}\cup \{(x, \infty) \mid x\in \mathbb{R}\}$.

Надо изучить: 
\begin{enumerate}
    \item окрестности точек
    \item замкнутые множества
    \item внутренности и замыкания
    \item является ли данная топология моделью классической логики
    \item связность
\end{enumerate}

Решение: 

\begin{enumerate}
    \item Окрестности точки $y$ это лучи $(x, \infty)$, где $x \geqslant y$
    \item Замкнутые множества имеют вид лучей $(-\infty, x]$
    \item Внутренность множества $X$ это луч $(x, +\infty)$, где $x = \sup \{x \mid (x,\infty)\subset X\}$. Замыкание находится аналогично.
    \item Рассотрим оценку пропозициональной переменной $A$ $(x, +\infty)$.
    $\mathbb{R} \setminus (x,+\infty) = (-\infty, x]$.  
    Тогда 
    \[ (\mathbb{R} \setminus A)^\circ = (-\infty, x]^\circ = \emptyset \]
    и очевидно $A \cup (\mathbb{R} \setminus A)^\circ \neq \mathbb{R}$. Эта топология не является моделью КИВ.
    \item Связность есть, потому что перечение любых непустых открытых множетсв непусто. 
\end{enumerate}
\end{document}