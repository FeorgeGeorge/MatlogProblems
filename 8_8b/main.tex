\documentclass[10pt]{article}
\pagestyle{plain}
\textwidth=15.5cm
\textheight=24.0cm
\oddsidemargin=0.3cm
\evensidemargin=0.5cm

\usepackage[utf8]{inputenc}
\usepackage[T2A]{fontenc}
\usepackage[russian]{babel}
\usepackage{amssymb,amsmath, amsthm}
\usepackage{graphicx}
\usepackage[a4paper,bmargin=2.5cm]{geometry}

\voffset=0pt
\headheight=0pt
\headsep=0pt

\begin{document}

\def\chap#1#2{\ \\ {\large\bf#1 \ | \ \tt\scshape#2} \par}

\ \vspace{-1cm}

{\bf
\ \\
\Large\centerline{\scshape Матлог 8.8b}
}\normalsize

\[\forall x. \forall y. x\in \omega \land y\in \omega \implies x^\prime = y^\prime \implies x = y\]

Рассмотрим $x,y$ --- ординалы (из $\omega$). 
Пусть $x^\prime = y^\prime$. 
Тогда $x \cup \{x\} = y \cup \{y\}$.
Есть несколько случаев.
\begin{enumerate}
    \item $x\in \{y\}$. Тогда немедленно $x = y$, ч.т.д.
    \item $y\in \{x\}$. Тогда сразу $y = x$, ч.т.д.
    \item Остаётся только $x\in y$ и $y\in x$. Вспомним определение транзитивного множества. 
    Множество $A$ называется транизитвным, если 
    \[ \forall a. \forall b. a\in b \land b \in A \implies a\in A \]
    Для элементов $z\in x$ в нашем случае это означает, что из $z \in x$ и $x \in y$ следует, что $z\in y$.
    Таким образом $z\in y$ для любого $z\in x$. 
    Отсюда $x\subset y$.
    Аналогично $y\subset x$ следует из $y\in x$.
    Тогда $x = y$, ч.т.д. 
\end{enumerate}
\end{document}