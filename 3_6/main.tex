\documentclass[10pt]{article}
\pagestyle{plain}
\textwidth=15.5cm
\textheight=24.0cm
\oddsidemargin=0.3cm
\evensidemargin=0.5cm

\usepackage[utf8]{inputenc}
\usepackage[T2A]{fontenc}
\usepackage[russian]{babel}
\usepackage{amssymb,amsmath, amsthm}
\usepackage{graphicx}
\usepackage[a4paper,bmargin=2.5cm]{geometry}

\voffset=0pt
\headheight=0pt
\headsep=0pt

\begin{document}

\def\chap#1#2{\ \\ {\large\bf#1 \ | \ \tt\scshape#2} \par}

\ \vspace{-1cm}

{\bf
\ \\
\Large\centerline{\scshape Матлог 3.6}
}\normalsize

Докажем, что открытые множества топологического пространства по отношению $\subset$ образуют импликативную решетку
Необходимо установить, что $A \cdot B$ (инфимум) и $A + B$ (супремум) корректно определены.

\[ A \cdot B = A\cap B\]
\[ A + B = A \cup B \]

\begin{proof}
    $A\cup B$ есть наибольшая нижняя грань множеств $A,B$ по включению.
    (если $X\subseteq A$ и $X \subseteq B$, то $X \subseteq A\cap B$).
    $A \cup B$ аналогично.  

    Докажем импилкативность.
    Необходимо найти $A\to B = \text{наибольшее } \{ C| A\cap C \subseteq B\}$. 
    Видно, что $C$ должно содержать $B$. 
    Так же заметим, что $C$ может содержать в себе все точки, которые не лежат в $A$.
    Тогда определим \[A\to B = (B \cup \neg A)^\circ \]
    Это искомое псевдодополнение. 
    Докажем это. 
    Пусть открытое $C$ такое, что $A\cap C \subseteq B$. 
    Покажем, что $C \subseteq B \cup \neg A$.
    Рассмотрим $c\in C$. 
    \begin{enumerate}
        \item $c\in A$. Тогда $c\in B \subset B\cup \neg A$;
        \item $c \in \neg A$. Тогда $c \in B \cup \neg A$.
    \end{enumerate}
    ч.т.д.

    
\end{proof}
\end{document}