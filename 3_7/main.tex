\documentclass[10pt]{article}
\pagestyle{plain}
\textwidth=15.5cm
\textheight=24.0cm
\oddsidemargin=0.3cm
\evensidemargin=0.5cm

\usepackage[utf8]{inputenc}
\usepackage[T2A]{fontenc}
\usepackage[russian]{babel}
\usepackage{amssymb,amsmath, amsthm}
\usepackage{graphicx}
\usepackage[a4paper,bmargin=2.5cm]{geometry}

\voffset=0pt
\headheight=0pt
\headsep=0pt

\begin{document}

\def\chap#1#2{\ \\ {\large\bf#1 \ | \ \tt\scshape#2} \par}

\ \vspace{-1cm}

{\bf
\ \\
\Large\centerline{\scshape Матлог 3.7}
}\normalsize

Рассмотим линейно-упорядоченное множество $(X, \leqslant)$.
Раз у нас все элементы $X$ сравнимы между собой, операции $a\cdot b$ и $a+b$ можно реализовать как
\[ a\cdot b = \min \{ a,b\} \]
\[ a+b = \max\{ a,b \} \] 
Это значит, что $(X, \leqslant)$ образует решетку.


Наличие $0,1$ зависит от $X$.
В $\mathbb{R}$ с $\pm \infty$ есть $0,1$.
В $\mathbb{Z}$ их нет.

Дистрибутивность.
Проверим \[ a+(bc) = (a+b)(a+c) \]
Другими словами, нужно проверить, что $\max (a, \min (b,c)) = \min (\max(a,b), \max(a,b))$.

\begin{enumerate}
    \item $a \geqslant \min(b,c)$. Тогда обе части равенства вычислятся как $a$.
    \item $a < \min(b,c)$. Тогда обе части вычислятся как $\min(b,c)$.
\end{enumerate}

Импликативности в общем случае нет (см. предыдущее задание). 
\end{document}