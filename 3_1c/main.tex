\documentclass[10pt]{article}
\pagestyle{plain}
\textwidth=15.5cm
\textheight=24.0cm
\oddsidemargin=0.3cm
\evensidemargin=0.5cm

\usepackage[utf8]{inputenc}
\usepackage[T2A]{fontenc}
\usepackage[russian]{babel}
\usepackage{amssymb,amsmath, amsthm}
\usepackage{graphicx}
\usepackage[a4paper,bmargin=2.5cm]{geometry}

\voffset=0pt
\headheight=0pt
\headsep=0pt

\begin{document}

\def\chap#1#2{\ \\ {\large\bf#1 \ | \ \tt\scshape#2} \par}

\ \vspace{-1cm}

{\bf
\ \\
\Large\centerline{\scshape Матлог 3.1c}
}\normalsize
\vspace{0.5cm}

Докажем, что внутренность и замыкание определены корретно как наибольшее открытое подмножество и наименьшее замкнутое надмножество соответственно.
Для замыкания всё будет аналогично (по закону де-Моргана), поэтому рассмотрим только внутренность.

\[ A^\circ = \bigcup_{B\in\Omega, B\subseteq A} B\]

\begin{proof}
    Это множество открыто как обьединение семейства открытых множеств.
    Очевидно, что $A^\circ$ и для любого открытого $C \subset A$ выполнено $C\subset A^\circ$. 
\end{proof}
\end{document}