\documentclass[10pt]{article}
\pagestyle{plain}
\textwidth=15.5cm
\textheight=24.0cm
\oddsidemargin=0.3cm
\evensidemargin=0.5cm

\usepackage[utf8]{inputenc}
\usepackage[T2A]{fontenc}
\usepackage[russian]{babel}
\usepackage{amssymb,amsmath, amsthm}
\usepackage{graphicx}
\usepackage[a4paper,bmargin=2.5cm]{geometry}

\voffset=0pt
\headheight=0pt
\headsep=0pt

\begin{document}

\def\chap#1#2{\ \\ {\large\bf#1 \ | \ \tt\scshape#2} \par}

\ \vspace{-1cm}

{\bf
\ \\
\Large\centerline{\scshape Матлог 6.4 e}
}\normalsize

Пусть $b >0$. Определим из $a = bp + q$ и $q < b$, чему равны $p, q$.
Определим $q = \frac{a}{b}$, где деление нацело опредлено индуктивно. 
\[ \frac{a}{b} = \begin{cases}
    (\frac{a - b}{b})^\prime, & a \geqslant b\\
    0, & a < b
\end{cases} \]

Тогда $q = a - \frac{a}{b}*b$.
Докажем сначала, что $\frac{a}{b} b \leqslant a$, т.е что 
\[ \exists p. a = \frac{a}{b}b + p\]

Докажем это ''модифицированной'' индукцией.
\[ (\forall a.\ a < b \implies P(a)) \land (\forall a.\ P(a)\implies P(a+b)), \text{влечёт } \forall a.\ P(a) \]

\begin{enumerate}
    \item $a < b$. $\frac{a}{b} = 0$, $p = a$.
    \item Пусть нашлось $p$: $a = \frac{a}{b}b + p$.
    Докажем утверждение для $a+b$. 
    \[\frac{a+b}{b}\cdot b = \frac{a}{b}b + b, \quad a+b = \frac{a+b}{b}b + p\]
\end{enumerate}

Эти $p,q$ будут единственными по построению.


\end{document}