\documentclass[10pt]{article}
\pagestyle{plain}
\textwidth=15.5cm
\textheight=24.0cm
\oddsidemargin=0.3cm
\evensidemargin=0.5cm

\usepackage[utf8]{inputenc}
\usepackage[T2A]{fontenc}
\usepackage[russian]{babel}
\usepackage{amssymb,amsmath, amsthm}
\usepackage{graphicx}
\usepackage[a4paper,bmargin=2.5cm]{geometry}

\voffset=0pt
\headheight=0pt
\headsep=0pt

\begin{document}

\def\chap#1#2{\ \\ {\large\bf#1 \ | \ \tt\scshape#2} \par}

\ \vspace{-1cm}

{\bf
\ \\
\Large\centerline{\scshape Матлог 4(b)}
}\normalsize

\[ (A \to B) \lor (B \to A)\]

Это решение опирается на закон об исключенном третьем, который тут не доказывается.
\begin{enumerate}
    \item $\alpha_1 \to (\alpha_n \to \alpha_1)$
    \item $\alpha_1 \to (\alpha_1 \to \alpha_2) \lor (\alpha_n \to \alpha_1)$
    \item $\neg \alpha_1 \to (\alpha_1 \to \alpha_2)$  (это доказанная схема $\neg \alpha \to \alpha \to \beta$)
    \item $\neg \alpha_1 \to (\alpha_1 \to \alpha_2) \lor (\alpha_n \to \alpha_1)$
    \item $(\alpha_1 \to (\alpha_1 \to \alpha_2) \lor (\alpha_n \to \alpha_1)) \to (\neg \alpha_1 \to (\alpha_1 \to \alpha_2) \lor (\alpha_n \to \alpha_1)) \to (\neg \alpha \lor \alpha) \to (\alpha_1 \to \alpha_2) \lor (\alpha_n \to \alpha_1)$
    \item $(\neg \alpha \lor \alpha) \to (\alpha_1 \to \alpha_2) \lor (\alpha_n \to \alpha_1)$
    \item $(\alpha_1 \to \alpha_2) \lor (\alpha_n \to \alpha_1)$
    \item[$6+n$] $((\alpha_1 \to \alpha_2) \lor (\alpha_n \to \alpha_1)) \to (\alpha_1 \to \alpha_2) \lor (\alpha_2 \to \alpha_3) \lor \ldots \lor (\alpha_n \to \alpha_1)$ 
    \item[$7+n$] $(\alpha_1 \to \alpha_2) \lor (\alpha_2 \to \alpha_3) \lor \ldots \lor (\alpha_n \to \alpha_1)$
\end{enumerate}


Если считать $\lor$ по правилу левой ассоциативности, то для формального обоснования двух последних шагов достаточно доказать следующую схему: 
\[ (\alpha \lor \beta) \lor \gamma \to (\alpha \lor \gamma) \lor \beta \]
\begin{enumerate}
    \item $\alpha \to \alpha \lor \gamma$ 
    \item $\alpha \lor \gamma \to (\alpha \lor \gamma) \lor \beta$
    \item $\alpha \to (\alpha \lor \gamma) \lor \beta$ (транизитивность $\to$ к предыдущим двум)
    \item $\gamma \to \alpha \lor \gamma$
    \item $\gamma \to (\alpha \lor \gamma) \lor \beta$ (транизитивность $\to$ к 4 и 2)
    \item $\beta \to (\alpha \lor \gamma) \lor \gamma$ 
    \item $(3) \lor (5) \lor (\alpha \lor \beta \to (\alpha \lor \gamma) \lor \beta)$.
    \item $\alpha \lor \gamma \to (\alpha \lor \gamma) \lor \beta$ (MP с 2,3 и предыдущим)
    \item $(8) \to (6) \to (\alpha \lor \beta) \lor \gamma \to (\alpha \lor \gamma) \to \beta$
    \item $(\alpha \lor \beta) \lor \gamma \to (\alpha \lor \gamma) \to \beta$
    \item $(\alpha \lor \gamma) \to \beta$ (MP с гипотезой)
\end{enumerate}
Теперь на каждом шаге процесса из 
\[ ((\alpha_1 \to \alpha_2) \lor (\alpha_2 \to \alpha_3) \lor \ldots (\alpha_i \to \alpha_{i+1})) \lor (\alpha_n \to \alpha_1)  \]
получается по схеме добавления $\lor$ 
\[ [((\alpha_1 \to \alpha_2) \lor (\alpha_2 \to \alpha_3) \lor \ldots (\alpha_i \to \alpha_{i+1})) \lor (\alpha_n \to \alpha_1)] \lor (\alpha_{i+1} \to \alpha_{i+2})\] 
и по новодоказанной схеме имеем 
\[ ((\alpha_1 \to \alpha_2) \lor (\alpha_2 \to \alpha_3) \lor \ldots (\alpha_i \to \alpha_{i+1}) \lor (\alpha_{i+1} \to \alpha_{i+2})) \lor (\alpha_n \to \alpha_1) \]
что и требовалось для перехода.

% { $\frac 12345$}

% \def\cmd#1{ #1 A \cmd}
% \let\cmd\cmd

% \cmd ABCDE
\end{document}