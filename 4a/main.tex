\documentclass[10pt]{article}

\pagestyle{plain}
\textwidth=15.5cm
\textheight=24.0cm
\oddsidemargin=0.3cm
\evensidemargin=0.5cm

\usepackage[utf8]{inputenc}
\usepackage[russian]{babel}
\usepackage{amssymb,amsmath, amsthm}
\usepackage{graphicx}
\usepackage[a4paper,bmargin=2.5cm]{geometry}


% новые окружения
\theoremstyle{definition}
\newtheorem{definition}{Определение} % окружение для определений
\newtheorem{theorem}{Теорема} % окружение для теорем
\newtheorem{corollary}{Следствие}[theorem] % окружение для следствий
\newtheorem{remark}{Замечание}

\newtheorem{lemma}{Лемма}[theorem] % окружение для лемм
\newtheorem{problem}{Задача} % окружение для задач

\voffset=0pt
\headheight=0pt
\headsep=0pt

\begin{document}

\def\chap#1#2{\ \\ {\large\bf#1 \ | \ \tt\scshape#2} \par}

\ \vspace{-1cm}

{\bf
\ \\
\Large\centerline{\scshape Матлог 4(a)}
}\normalsize

\[ (A \to B) \lor (B \to A)\]

Это решение опирается на закон об исключенном третьем, который тут не доказывается.
\begin{enumerate}
    \item $A \to (B \to A)$ (схема 1)
    \item $A \to (A\to B) \lor (B \to A)$ (добавление $\lor$)
    \item $\neg A \to (A \to B)$ (выводится из теоремы о дедукции и задачи 1e)
    \item $\neg A \to (A\to B) \lor (B \to A)$ (добавление $\lor$)
    \item $(A \to (A\to B) \lor (B \to A)) \to (\neg A \to (A \to B) \lor (B \to A)) \to ((A \lor \neg A) \to ((A \to B) \lor (B \to A)))$
    \item $(A \lor \neg A) \to ((A \to B) \lor (B \to A))$
    \item $A \lor \neg A$
    \item $(A \to B) \lor (B \to A)$ (MP с законом исключенного третьего) 
 \end{enumerate}

\end{document}