\documentclass[10pt]{article}
\pagestyle{plain}
\textwidth=15.5cm
\textheight=24.0cm
\oddsidemargin=0.3cm
\evensidemargin=0.5cm

\usepackage[utf8]{inputenc}
\usepackage[T2A]{fontenc}
\usepackage[russian]{babel}
\usepackage{amssymb,amsmath, amsthm}
\usepackage{graphicx}
\usepackage[a4paper,bmargin=2.5cm]{geometry}

\voffset=0pt
\headheight=0pt
\headsep=0pt

\begin{document}

\def\chap#1#2{\ \\ {\large\bf#1 \ | \ \tt\scshape#2} \par}

\ \vspace{-1cm}

{\bf
\ \\
\Large\centerline{\scshape Матлог 6.5c}
}\normalsize



Доказательство в формальной арифметике $\forall p. \forall q. p\cdot q = 0 \implies p = 0 \lor q = 0$.
В КИВ выполнено $p\cdot q = 0 \implies p = 0 \lor q = 0 \equiv p \cdot q = 0 \implies \neg p = 0 \implies q = 0$
Поэтому докажем сначала 
\[ p \cdot q = 0, \neg p = 0 \vdash q = 0,\]
(и потом повесим $\forall p. \forall q. \ldots$)

\begin{enumerate}
    \item $p\cdot q = 0$
    \item $\neg p = 0$
    \item $\forall \neg p = 0 \implies \exists s. s^\prime = p$
    \item $\forall \neg p = 0 \implies \exists s. s^\prime = p \implies \neg p = 0 \implies \exists s. s^\prime = p$
    \item $\neg p = 0 \implies \exists q. q^\prime = p$
    \item $\exists q. q^\prime = p$
    \item[] Далее следует вывод, в котором для предыдущих утверждений выполнена подстановка $[p = s^\prime]$. Теперь докажем такую теорему индукцией по $q$: 
    \[ P = s^\prime q = 0 \implies q = 0\] 
    
    % \item  $s^\prime 0 = 0$ (аксиома A7)
    \item $0 = 0$
    \item $0 = 0 \implies (s^\prime 0 = 0) \implies 0 = 0$
    \item $(s^\prime \cdot 0 = 0) \implies 0 = 0$ --- $P[q=0]$
    \item $P[q = 0] \land (\forall q. P \implies P[q = q^\prime]) \implies P$ (привило индукции)
    \item[] докажем переход $P \implies P[q = q^\prime]$.
    \item $s^\prime \cdot q^\prime = s^\prime + s^\prime q$
    \item $s^\prime + s^\prime q = (s + s^\prime q)^\prime$
    \item $s^\prime \cdot q^\prime = s^\prime + s^\prime q \implies s^\prime + s^\prime q = (s + s^\prime q)^\prime \implies s^\prime q^\prime = (s + s^\prime q)^\prime$ (А1 и свойства ''='', которые из неё следуют) 
    \item $s^\prime q^\prime = (s + s^\prime q)^\prime$ 
    \item $\neg (s + s^\prime q)^\prime = 0$
    \item [] тогда следует (неформально)
    \item $\neg s^\prime q^\prime = 0$
    \item $s^\prime q^\prime = 0 \implies (\neg s^\prime q^\prime 0) \implies (q = 0)$ (аксиома 10 в КИВ)
    \item[] $\ldots$ тогда
    \item $s^\prime q^\prime = 0 \implies q = 0$
    \item $P \implies s^\prime q^\prime = 0 \implies q = 0$    
\end{enumerate}
\end{document}